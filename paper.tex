\documentclass{article}

\author{Suman Chapai}
\title{Rate Limiter - CS 492 Proposal}

%
% Packages
%
\usepackage[
backend=biber,
style=alphabetic,
sorting=ynt
]{biblatex}
\addbibresource{ref.bib}
\usepackage{amssymb}
\usepackage{array}
\usepackage{hyperref}




%
% Macros
%
\newcommand*{\Z}[0]{\mathbb{Z}}



\begin{document}
\maketitle



%
% Intro
%
\section{Introduction}
We propose building a programmable rate limiter as part of CS 492 to be
completed in the fall of 2023. The rate limiter will accept incoming connections
and forward requests to one or more servers if the requests get filtered through
the specified filtering configuration (based on Application and IP layer).
Packets corresponding to requests that don't pass through the filter will be
dropped. In this document, we outline proposed functionalities and timeline.


%
% Definitions
%
\section{Definitions}
\begin{enumerate}
    \item
          \textbf{Filter} and other tense forms of the verb connote whether a
          request is filtered through (meaning accepted) or filtered out
          (meaning rejected) by the rate limiter.
    \item
          \textbf{Ursa} roughly pronounced as \textit{woor-sa} may be used
          interchangeably with the noun phrase \textbf{rate limiter} and denotes
          the proposed rate limiter server described herein.
    \item
          \textbf{Upstream} denotes the server(s) sitting behind the rate limiter.
          Envoy \cite{envoy} was taken as a reference for this terminology.
\end{enumerate}

%
% Non Functional Requirements
%
\section{Functional Requirements}


\subsection{HTTP Filtering}
\begin{enumerate}
    \item
          Read HTTP headers, modify/add HTTP headers before sending request upstream.
    \item
          Retrieve the number of of request based on given parameter for example
          HTTP header field or IPv4 or IPv6 address in the last $x$ seconds where
          $0 < x < k$ for some $k \in \Z$.
    \item
          Allow to define configuration specifications above in a configuration file.
    \item
          Filter based on configuration.
    \item
          Log requests and their filtering status.
\end{enumerate}


%
% Non Functional Requirements
%
\section{Non-Functional Requirements}
\begin{enumerate}
    \item
          Construct a type-safe domain language to define Ursa configuration.
    \item
          In addition to in-memory database that Ursa uses by default, allow users to
          specify a persistent database.
    \item
          Expose API to update configuration used by Ursa.
\end{enumerate}

%
% Timeline
%
\section{Timeline}
\begin{tabular}{||m{2cm} | m{10cm} ||}
    \hline
    Deadline     & Goal                                                                                           \\
    \hline \hline
    Apr 30       & Finalize language, submit proposal draft                                                       \\
    \hline
    May 30       & Complete minimal Go server to receive/forward requests (in-memory DB, hardcoded configuration) \\
    \hline
    June 7       & Make it possible to use yaml based configuration                                               \\
    \hline
    June 30      & Explore possibilities of domain specific language to define configuration                      \\
    \hline
    July 30      & Complete server functionalities                                                                \\
    \hline
    August 30    & Write integration tests                                                                        \\
    \hline
    September 7  & Support docker. Publish docker image running Ursa.                                                                   \\
    \hline
    September 30 & Setup load testing environment to note benchmarks                                           \\
    \hline
    October 30   & Work on domain specific language for defining configuration                                             \\
    \hline
    November 30  & Create and publish documentation                                                               \\
    \hline
\end{tabular}


%
% 
%
\section{Technical Specifications}
We plan on writing the Ursa server in \textit{Go}. Code will be hosted on
in the \href{https://github.com/ursaserver}{usraserver github}.

%
% Related Work
%
\section{Related Work}
A rate limiter server is a part of API gateway of most, if not every, large
company as it is one of the primary mitigation measures against DOS and DDOS
attacks.

We have found two systems in the internet today that we will use as examples
when building Ursa: Envoy \cite{envoy} is a proxy server written in \textit{C++}
that has tons of features including Application layer and IP layer rate
limiting. Another example we will look at is Varnish \cite{varnish} that is a
caching proxy whose caching behavior can be defined using a Varnish specific
configuration language called Varnish Configuration Language (VCL).  We will
look at VCL for reference when implementing the configuration language for Ursa.

\printbibliography

\end{document}
